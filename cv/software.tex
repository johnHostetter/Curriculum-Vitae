\cvsection{Software Products {\footnotesize (\dag $\,$ for Owner \& Sole Developer)}}

\begin{cvpubs}
\indent
    \cvpub{\textbf{PySoft\textsuperscript{\dag} \textit{(closed-source)}:} {\footnotesize A closed-source Python Soft Computing library capable of \textit{dynamically} self-organizing neuro-fuzzy networks in a wide variety of conditions, such as supervised or reinforcement learning. The library is built upon a graph theory (\texttt{igraph}) and \texttt{PyTorch} backend which may display its internal workings through animations using \texttt{maninlib}. PySoft also contains in-house implementations of algorithms such as linguistic summarization, quantitative temporal transaction database mining, categorical learning-induced partitioning, evolving clustering methods, and more. It contains wrappers, interfaces and callbacks to work alongside \texttt{d3rlpy}, \texttt{sb3}, \texttt{torchrl}, \texttt{wandb}, \texttt{ClearML}, \texttt{ViZDOOM}, and many others.}}
    \\
    \cvpub{\textbf{Soft-Computing\textsuperscript{\dag} \textit{(open-source)}:} {\footnotesize A Python repository which hosts Numpy implementations rough set theory, genetic fuzzy systems, All-Permutations Fuzzy Rule Base, Fuzzy Q-Learning, and more. \href{https://github.com/johnHostetter/Soft-Computing}{https://github.com/johnHostetter/Soft-Computing}}}
    \\
    \cvpub{\textbf{Neuro-Fuzzy-Networks\textsuperscript{\dag} \textit{(open-source)}:} {\footnotesize Implemented an adaptive neuro-fuzzy inference system (HyFIS) and a self-adaptive fuzzy inference network (SaFIN) in Numpy. \href{https://github.com/johnHostetter/Neuro-Fuzzy-Networks}{https://github.com/johnHostetter/Neuro-Fuzzy-Networks}}}
    \\
    \cvpub{\textbf{Fuzzy Reinforcement Learning Repositories\textsuperscript{\dag} \textit{(open-source)}:} {\footnotesize Wrote various algorithms pertaining to Fuzzy Reinforcement Learning in Numpy such as GARIC (\href{https://github.com/johnHostetter/GARIC}{https://github.com/johnHostetter/GARIC}), GPFRL (\href{https://github.com/johnHostetter/GPFRL}{https://github.com/johnHostetter/GPFRL}) and Kohonen's Self Organizing Map (\href{https://github.com/johnHostetter/Kohonen}{https://github.com/johnHostetter/Kohonen}).}}
    \\
    \cvpub{\textbf{PolicyPrep\textsuperscript{\dag} \textit{(open-source)}:} {\footnotesize Owner and sole developer of a configurable pipeline that automates experiment setup for real-world studies involved with offline reinforcement learning in the educational domain. It downloads data from Google Drive, pre-processes and resolves conflicts in the data, infers immediate rewards from delayed rewards, induces a variety of policies via different algorithms simultaneously, and performs offline off-policy evaluation. This project has saved each lab member approximately 1 to 2 month's worth of labor per year. \href{https://github.com/johnHostetter/PolicyPrep}{https://github.com/johnHostetter/PolicyPrep}}}
    \\
    \noindent
    \cvpub{\textbf{manim-beamer\textsuperscript{\dag} \textit{(open-source)}:} {\footnotesize Python emulation of LaTeX beamer within manim-slides, an interactive tool for live presentations using a math animation library called Manim (community edition). \href{https://github.com/johnHostetter/manim-beamer}{https://github.com/johnHostetter/manim-beamer}}}
    \\
    \noindent
    \cvpub{\textbf{manim-timeline\textsuperscript{\dag} \textit{(open-source)}:} {\footnotesize A more fun and interactive mode of presentation compared to PowerPoint or LaTeX to review literature and propose new material; allows rapid but clear communication by seamlessly integrating history, quotes, publications, and relevant demos by gradually building upon a visual timeline. \href{https://github.com/johnHostetter/manim-timeline}{https://github.com/johnHostetter/manim-timeline}}}
    \\
    \noindent
    \cvpub{\textbf{FCQL Demo\textsuperscript{\dag} \textit{(open-source \& published)}:} {\footnotesize PyTorch demo of Fuzzy Conservative Q-Learning (FCQL) using the systematic design process outlined in \href{https://dl.acm.org/doi/10.5555/3545946.3598770}{AAMAS 2023}. Available on \href{https://github.com/johnHostetter/AAMAS-2023-FCQL}{GitHub} and \href{https://zenodo.org/records/7668308}{Zenodo}: \texttt{johnHostetter/AAMAS-2023-FCQL: First release (v1.0.0). Zenodo. https://doi.org/10.5281/zenodo.7668308}}}
    \\
    \noindent
    \cvpub{\textbf{HepiusApp (formerly OsteoApp) \textit{(closed-source \& published)}}: {\footnotesize A mobile app (built with C\# \& Xamarin) to research how the older population uses smartphones to make healthcare decisions regarding Osteoporosis prevention and treatment. Published publicly to the Apple App Store, and for internal beta testing to Google Play Store. The special intellectual property of HepiusApp was transferred to Penn State Research Foundation on October 29, 2019. Clinical trials on patients using the app were supervised by Dr. Russell Kirkscey, Dr. Edward Fox, and Dr. Hien Nguyen. Its development led to the publication of the following articles:}} 
    \begin{enumerate}
        \item     \href{https://pure.psu.edu/en/publications/development-and-patient-user-experience-evaluation-of-an-mhealth-}{{\texttt{
    Kirkscey, R. (2021). Development and Patient User Experience Evaluation of an mHealth Informational\\App for Osteoporosis. International Journal of Human–Computer Interaction, 38(8), 707–718. 
    }}}
    \item     \href{https://pure.psu.edu/en/publications/mhealth-apps-for-older-adults-a-method-for-development-and-user-e}{{\texttt{
    Kirkscey, R. (2021). mHealth Apps for Older Adults: A Method for Development and User Experience\\Design Evaluation. Journal of Technical Writing and Communication, 51(2), 199-217. 
    %https://doi.org/10.1177/0047281620907939
    }}}
    \end{enumerate}
\end{cvpubs}

